%\documentclass[12pt,a4paper]{scrartcl}
\documentclass[12pt,a4paper]{article}

\makeatletter % Technical doc - START

\usepackage[utf8]{inputenc}
\usepackage[T1]{fontenc}
\usepackage{ucs}

\usepackage[french]{babel,varioref}

\usepackage[top=2cm, bottom=2cm, left=1.5cm, right=1.5cm]{geometry}
\usepackage{enumitem}

\usepackage{multicol}

\usepackage{makecell}

\usepackage{color}
\usepackage{hyperref}
\hypersetup{
    colorlinks,
    citecolor=black,
    filecolor=black,
    linkcolor=black,
    urlcolor=black
}

\usepackage{amsthm}

\usepackage{tcolorbox}
\tcbuselibrary{listingsutf8}

\usepackage{ifplatform}

\usepackage{ifthen}

\usepackage{cbdevtool}



% MISC

\newtcblisting{latexex}{%
	sharp corners,%
	left=1mm, right=1mm,%
	bottom=1mm, top=1mm,%
	colupper=red!75!blue,%
	listing side text
}

\newtcblisting{latexex-flat}{%
	sharp corners,%
	left=1mm, right=1mm,%
	bottom=1mm, top=1mm,%
	colupper=red!75!blue,%
}

\newtcblisting{latexex-alone}{%
	sharp corners,%
	left=1mm, right=1mm,%
	bottom=1mm, top=1mm,%
	colupper=red!75!blue,%
	listing only
}


\newcommand\env[1]{\texttt{#1}}
\newcommand\macro[1]{\env{\textbackslash{}#1}}



\setlength{\parindent}{0cm}
\setlist{noitemsep}

\theoremstyle{definition}
\newtheorem*{remark}{Remarque}

\usepackage[raggedright]{titlesec}

\titleformat{\paragraph}[hang]{\normalfont\normalsize\bfseries}{\theparagraph}{1em}{}
\titlespacing*{\paragraph}{0pt}{3.25ex plus 1ex minus .2ex}{0.5em}


\newcommand\separation{
	\medskip
	\hfill\rule{0.5\textwidth}{0.75pt}\hfill
	\medskip
}


\newcommand\extraspace{
	\vspace{0.25em}
}


\newcommand\whyprefix[2]{%
	\textbf{\prefix{#1}}-#2%
}

\newcommand\mwhyprefix[2]{%
	\texttt{#1 = #1-#2}%
}

\newcommand\prefix[1]{%
	\texttt{#1}%
}


\newcommand\inenglish{\@ifstar{\@inenglish@star}{\@inenglish@no@star}}

\newcommand\@inenglish@star[1]{%
	\emph{\og #1 \fg}%
}

\newcommand\@inenglish@no@star[1]{%
	\@inenglish@star{#1} en anglais%
}


\newcommand\ascii{\texttt{ASCII}}


% Example
\newcounter{paraexample}[subsubsection]

\newcommand\@newexample@abstract[2]{%
	\paragraph{%
		#1%
		\if\relax\detokenize{#2}\relax\else {} -- #2\fi%
	}%
}



\newcommand\newparaexample{\@ifstar{\@newparaexample@star}{\@newparaexample@no@star}}

\newcommand\@newparaexample@no@star[1]{%
	\refstepcounter{paraexample}%
	\@newexample@abstract{Exemple \theparaexample}{#1}%
}

\newcommand\@newparaexample@star[1]{%
	\@newexample@abstract{Exemple}{#1}%
}


% Change log
\newcommand\topic{\@ifstar{\@topic@star}{\@topic@no@star}}

\newcommand\@topic@no@star[1]{%
	\textbf{\textsc{#1}.}%
}

\newcommand\@topic@star[1]{%
	\textbf{\textsc{#1} :}%
}

\makeatother % Technical doc - END


\usepackage{tnsana}


\begin{document}

\renewcommand\labelitemi{\raisebox{0.125em}{\tiny\textbullet}}
\renewcommand{\labelitemii}{---}

\title{ %
	Le package \texttt{tnsana}:\\%
	de l'analyse élémentaire%
	\\%
	{\footnotesize Code source disponible sur \url{https://github.com/typensee-latex/tnsana.git}.}%
	\\%
    {\footnotesize Version \texttt{0.2.0-beta} développée et testée sur \macosxname{}.}%
}
\author{Christophe BAL}
\date{2020-07-15}

\maketitle


\vspace{2em}

\hrule

\tableofcontents

\vspace{1.5em}

\hrule

\newpage

\section{Introduction}

Le package \verb+tnsana+ propose des macros pour écrire de l'analyse mathématique basique via un codage sémantique simple.

\begin{remark}
	Ce package s'appuie sur \verb+tnscom+ disponible sur \url{https://github.com/typensee-latex/tnscom.git}.
\end{remark}
\section{Constantes et paramètres}

\subsection{Constantes classiques}

\paragraph{La liste complète}

% List of classical constants - START

\begin{latexex}
$\ggamma$ , $\ppi$ , $\ttau$ ,
$\ee$ , $\ii$ , $\jj$ 
et $\kk$ où $\ttau = 2 \ppi$
\end{latexex}

% List of classical constants - END


\begin{remark}
	Faites attention car \verb+{\Large $\ppi \neq \pi$}+ produit {\Large $\ppi \neq \pi$}. Comme vous le constatez, les symboles ne sont pas identiques. Ceci est vraie pour toutes les constantes grecques.
\end{remark}


% ---------------------- %


\subsection{Constantes latines personnelles}

La macro \macro{param} est surtout là pour une utilisation pédagogique.

\begin{latexex}
$\param{a} x^2 + \param{b} x + \param{c}$
ou
$a x^2 + b x + c$
\end{latexex}


% ---------------------- %
\section{Une variable \og symbolique \fg{}}

Le package \verb+tnscom+ propose les macros \macro{symvar} et \macro{symvar*} qui produisent les symboles $\symvar$ et $\symvar*$ . Le nom utilisé vient de \prefix{symvar} pour \whyprefix{symb}{olic} \whyprefix{var}{iable} soit \inenglish{variable symbolique}.
Ces symboles sont utiles pour indiquer un argument symboliquement sans faire référence précisément à une ou des variables nommées.


% ---------------------- %
\section{La fonction valeur absolue}

\newparaexample*{}

\begin{latexex}
$\abs{2}$ ,
$\abs {\dfrac{3}{5}}$ ou
$\abs*{\dfrac{3}{5}}$
\end{latexex}


\begin{remark}
	Le code \LaTeX{} vient directement de ce poste : \url{https://tex.stackexchange.com/a/43009/6880}.
\end{remark}


% ---------------------- %
\section{Fonctions nommées spéciales}

\subsection{Sans paramètre}

Quelques fonctions nommées supplémentaires où le \prefix{f} dans \prefix{fch} est pour \whyprefix{f}{rench} soit \inenglish{français} \emph{(ce choix a été fait pour éviter des incompatibilités avec quelques autres packages)}. La liste complète des fonctions nommées est donnée un peu plus bas dans la section  \ref{tnsana-all-named-functions}.

\begin{latexex}
$\fch x = \cosh x$ ou
$\lg x$
\end{latexex}


% ---------------------- %


\subsection{Avec un paramètre}

Pour le moment il y a juste deux fonctions avec un paramètre. Les voici.

\begin{latexex}
$\logb{2} x = \lg x$ ou
$\expb{6} y = 6^y$
\end{latexex}


% ---------------------- %


\subsection{Toutes les fonctions nommées en plus} \label{tnsana-all-named-functions}

\vspace{-1em}

\begin{multicols}{2}
% Table of all - START
    \verb+acos+ : $\acos\dots$

    \verb+asin+ : $\asin\dots$

    \verb+atan+ : $\atan\dots$

    \verb+arccosh+ : $\arccosh\dots$

    \verb+arcsinh+ : $\arcsinh\dots$

    \verb+arctanh+ : $\arctanh\dots$

    \verb+acosh+ : $\acosh\dots$

    \verb+asinh+ : $\asinh\dots$

    \verb+atanh+ : $\atanh\dots$

    \verb+fch+ : $\fch\dots$

    \verb+fsh+ : $\fsh\dots$

    \verb+fth+ : $\fth\dots$

    \verb+afch+ : $\afch\dots$

    \verb+afsh+ : $\afsh\dots$

    \verb+afth+ : $\afth\dots$

    \verb+expb{p}+ : $\expb{p}\dots$

    \verb+logb{p}+ : $\logb{p}\dots$
% Table of all - END
\end{multicols}
\section{Calcul différentiel}

\subsection{\texorpdfstring{Les opérateurs $\pp{}$ et $\dd{}$}%
                           {Les opérateurs "d rond" et "d droit"}}

Voici deux opérateurs utiles aussi bien pour du calcul différentiel que du calcul intégral. 

\begin{latexex}
$\dd{t} = \dd[1]{t}$ ou $\dd[n]{x}$

$\pp{t} = \pp[1]{t}$ ou $\pp[n]{x}$
\end{latexex}


% ---------------------- %


\subsection{Dérivations totales d'une fonction -- Version longue mais polymorphe}

\newparaexample{Différentes écritures possibles}

La macro \macro{der} est stricte du point de vue sémantique car on doit lui fournir la fonction, l'ordre de dérivation et la variable de dérivation
\emph{(voir la section \ref{tnsana-short-der} qui présente la macro \macro{sder} permettant une rédaction efficace pour obtenir $\sder[e]{f}{1}$ ou $\sder{f}{1}$)}.
Voici plusieurs mises en forme faciles à taper via l'option de \macro{der}.
Attention bien entendu à n'utiliser l'option par défaut \prefix{u} ou l'option \prefix{d} qu'avec un ordre de dérivation de valeur naturelle connue !

\begin{latexex}
 $\der   {f}{3}{x}
= \der[e]{f}{3}{x}
= \der[d]{f}{3}{x}$

 $\der[i] {u}{k}{x}
= \der[f] {u}{k}{x}
= \der[sf]{u}{k}{x}$
\end{latexex}


On peut aussi ajouter autour de la fonction des parenthèses extensibles ou non sauf pour le mode \prefix{d} \emph{(voir juste après)}.
Ci-dessous on montre au passage une écriture du type \emph{\og opérateur fonctionnel \fg}.

\begin{latexex}
 $\der[osf,sp]{\frac{1}{2}  uv}{k}{x}
= \der[of,p]  {\dfrac{1}{2} uv}{k}{x}$
\end{latexex}


Avec l'option \prefix{d} les parenthèses étant sans utilité on obtient une autre mise en forme. En voici un exemple. Notez au passage qu'ici \prefix{sp} n'apporte rien de nouveau.

\begin{latexex}
 $\der[d,p] {\frac{1}{2}  uv}{2}{x}
= \der[d,sp]{\dfrac{1}{2} uv}{2}{x}$
\end{latexex}


\begin{remark}
	Expliquons les valeurs des options.
	\begin{enumerate}
		\item \prefix{u}, la valeur par défaut, est pour \whyprefix{u}{suel} soit l'écriture avec les primes. Cette option ne marchera pas avec un nombre symbolique de dérivations. 

		\item \prefix{e} est pour \whyprefix{e}{xposant}.

		\item \prefix{d} est pour \whyprefix{d}{ot} soit \inenglish{point}.

		\smallskip
		\item \prefix{i} est pour \whyprefix{i}{ndice}.

		\item \prefix{f} est pour \whyprefix{f}{raction} avec aussi \prefix{sf} pour une écriture réduite où \prefix{s} est pour \whyprefix{s}{mall} soit \inenglish{petit}.

		\item \prefix{of} et \prefix{osf} utilisent le préfixe \prefix{o} pour \whyprefix{o}{pérateur}.
		
		\smallskip
		\item \prefix{p} est pour \whyprefix{p}{arenthèse} : dans ce cas les parenthèses seront extensibles. Le fonctionnement est différent avec l'option \prefix{d} comme nous l'avons vu avant.

		\item \prefix{sp} est pour des parenthèses non extensibles. Là aussi le fonctionnement est différent avec l'option \prefix{d}.
	\end{enumerate}
\end{remark}


% ---------------------- %


\newparaexample{Pas de uns inutiles}

\begin{latexex}
 $\der[i ]{u}{1}{x}
= \der[f ]{u}{1}{x}
= \der[sf]{u}{1}{x}
= \der[of]{u}{1}{x}$
\end{latexex}


\begin{remark}
	Voici comment forcer les exposants $1$ si besoin. Fonctionnel mais un peu moche...

	\begin{latexex}
 $\der[i ]{u}{\,\!1}{x}
= \der[f ]{u}{\,\!1}{x}
= \der[sf]{u}{\,\!1}{x}
= \der[of]{u}{\,\!1}{x}$
\end{latexex}
\end{remark}


% ---------------------- %


\subsection{Dérivations totales d'une fonction -- Version courte pour les écritures standard} \label{tnsana-short-der}

Dans l'exemple suivant le code manque de sémantique car on n'indique pas la variable de dérivation.
Ceci étant dit à l'usage la macro \macro{sder} rend de grands services.
Ici le préfixe \prefix{s} est pour \whyprefix{s}{imple} voire \whyprefix{s}{impliste}...
Voici des exemples où de nouveau l'option par défaut \prefix{u} et l'option \prefix{d} ne seront fonctionnelles qu'avec un ordre de dérivation de valeur naturelle connue !


\newparaexample{}

\begin{latexex}
 $\sder{f}{1} = \der{f}{1}{x}$

 $\sder{f}{1}
= \sder[e]{f}{1}
= \sder[d]{f}{1}$
\end{latexex}


\newparaexample{}

\begin{latexex}
 $\sder[sp ]{\dfrac{1}{2} uv}{2}
= \sder[e,p]{\dfrac{1}{2} uv}{2}
= \sder[d,p]{\dfrac{1}{2} uv}{2}$
\end{latexex}


\begin{remark}
	Ici les seules options disponibles sont \prefix{u}, \prefix{e}, \prefix{p} et \prefix{sp}.
\end{remark}


% ---------------------- %


\subsection{L'opérateur de dérivation totale}

Ce qui suit peut rendre service au niveau universitaire.
Les options possibles sont \verb+f+, valeur par défaut, \verb+sf+ et \verb+i+ avec les mêmes significations que pour la macro \macro{der}.

\begin{latexex}
 $\derope    {k}{x}
= \derope[sf]{k}{x}
= \derope[i] {k}{x}$
\end{latexex}


Ici non plus il n'y a pas de uns inutiles.

\begin{latexex}
 $\derope    {1}{x}
= \derope[sf]{1}{x}
= \derope[i] {1}{x}$
\end{latexex}


% ---------------------- %
%\section{Calcul différentiel}

\subsection{Dérivations partielles}

\newparaexample{Différentes écritures possibles}

La macro \macro{pder}
\footnote{
	\macro{partial} existe déjà pour obtenir $\partial$.
}
avec \prefix{p} pour \whyprefix{p}{artielle}
permet de rédiger des dérivées partielles en utilisant facilement plusieurs mises en forme via une option qui vaut \verb+f+ par défaut.
Cette macro attend une fonction, les dérivées partielles effectuées et l'ordre total de dérivation.
Voici les deux types de mise en forme où vous noterez comment \verb+x | y^2+ est interprété.

\begin{latexex}
 $\pder    {f}{x | y^2}{3}
= \pder[sf]{f}{x | y^2}{3}$
ou
 $\pder[i] {f}{x | y^2}{3}$
\end{latexex}


On peut aussi ajouter autour de la fonction des parenthèses extensibles ou non.
Ci-dessous on montre aussi une écriture du type \emph{\og opérateur fonctionnel \fg}.

\begin{latexex}
 $\pder[of] {f}{x | y^2}{}
= \pder[osf]{f}{x | y^2}{}$
ou
 $\pder[i,sp]{u + v}{x | y^2}{}$
\end{latexex}


\begin{remark}
	Les options disponibles sont 
	\verb+f+, \verb+sf+, \verb+of+, \verb+osf+, \verb+i+, \verb+p+ et \verb+sp+  
	avec des significations similaires à celles pour la macro \macro{der}.
\end{remark}


% ---------------------- %


\newparaexample{Pas de uns inutiles}

\begin{latexex}
 $\pder    {u}{x}{1}
= \pder[sf]{u}{x}{1}
= \pder[i] {u}{x}{1}$
\end{latexex}


% ---------------------- %


\subsection{L'opérateur de dérivation partielle}

Ce qui suit peut rendre service au niveau universitaire.
Les options possibles sont \verb+f+, valeur par défaut, \verb+sf+ et \verb+i+ avec les mêmes significations que pour la macro \macro{pder}.

\begin{latexex}
 $\pderope    {x | y^2}{3}
= \pderope[sf]{x | y^2}{3}
= \pderope[i] {x | y^2}{3}$
\end{latexex}


% ---------------------- %
% \section{Analysis}

\section{Tableaux de variation et de signe}

\subsection{Les bases de \texttt{tkz-tab}}

\paragraph{Comment ça marche ?}

Pour les tableaux de variation et de signe non décorés, tout le boulot est fait par le package \verb+tkz-tab+.
Ce package est utilisé via le réglage \verb+\tkzTabSetup[arrowstyle = triangle 60]+ afin d'obtenir des pointes de flèche plus visibles.

\medskip

Nous donnons quelques exemples classiques d'utilisation proches ou identiques de certains proposés dans la documentation de \verb+tkz-tab+ \emph{(les codes ont été mis en forme pour faciliter la compréhension de la syntaxe à suivre)}.
Reportez vous à la documentation de \verb+tkz-tab+ pour des compléments d'information : vous y trouverez des réglages très fins.


% ---------------------- %


\newparaexample{Avec des signes}

\begin{latexex-flat}
\begin{tikzpicture}
    \tkzTabInit{
        $x$       / 0.75 , % Facteur d'échelle de 0.75 pour la hauteur de la 1er ligne.
        $\cos(x)$ / 1      % Facteur d'échelle de  1   pour la hauteur de la 2e ligne.
    }{
                $0$     , $\frac{\pi}{2}$     , $\pi$  % Valeurs utiles de x.
    }
    \tkzTabLine{    , + , z               , - ,      } % Signes et zéro.
\end{tikzpicture}
\end{latexex-flat}


% ---------------------- %


\newparaexample{Avec des variations (sans cadre)}

\begin{latexex-flat}
\begin{tikzpicture}
    \tkzTabInit[nocadre]{
        $x$    / 0.75 ,
        $f(x)$ / 1.5
    }{
               $-\infty$ , $p$        , $+\infty$
    }
    \tkzTabVar{+ /       , - / $f(p)$ , + /      } % Variations via position / valeur.
\end{tikzpicture}
\end{latexex-flat}


% ---------------------- %


\newparaexample{Variations via une dérivée (sans cadre)}

\begin{latexex-flat}
\begin{tikzpicture}
    \tkzTabInit[nocadre]{
        $x$       / 0.75 ,
        $\cos(x)$ / 1    ,
        $\sin(x)$ / 1.5
    }{
                $0$     , $\frac{\pi}{2}$     , $\pi$
    }
    \tkzTabLine{    , + , z               , - ,      }
    \tkzTabVar {- / 0   , + / 1               , - / 0}
\end{tikzpicture}
\end{latexex-flat}


% ---------------------- %


\newparaexample{Une image intermédiaire avec une seule flèche}

\begin{latexex-flat}
\begin{tikzpicture}
    \tkzTabInit{
        $x$     / 0.75 ,
        $3 x^2$ / 1    ,
        $x^3$   / 1.5
    }{
                $-\infty$         , $0$     , $+\infty$
    }
    \tkzTabLine{              , + , 0   , + ,              }
    \tkzTabVar {- / $-\infty$     , R       , + / $+\infty$}
    %
    \tkzTabIma{1}{3}{2} % Position entre les 1re et 3e valeurs puis rang relatif.
              {$0$}     % Valeur de l'image.
\end{tikzpicture}
\end{latexex-flat}


% ---------------------- %


\newparaexample{Valeurs interdites et valeurs supplémentaires}

\begin{latexex-flat}
\begin{tikzpicture}
    \tkzTabInit[espcl = 6]{    % Largeur entre les valeurs du tableau.
        $x$            / 0.75 ,
        $\frac{1}{x}$ / 1.25 ,
        $\ln$          / 1.75
    }{
                $0$                , $+\infty$
    }
    \tkzTabLine{d              , + ,              }
    \tkzTabVar {D- / $-\infty$     , + / $+\infty$}
    %
    \tkzTabVal{1}{2}{0.35} % Position entre les 1re et 2e valeurs puis en proportion.
              {1}{0}       % x_1 et f(x_1)
    \tkzTabVal{1}{2}{0.65} % Position entre les 1re et 2e valeurs puis en proportion.
              {$\ee$}{1}   % x_2 et f(x_2)
\end{tikzpicture}
\end{latexex-flat}

Voici un autre exemple pour comprendre comment utiliser \macro{tkzTabVal} avec en plus l'option \verb+draw+ qui peut rendre service.

\begin{latexex-flat}
\begin{tikzpicture}
    \tkzTabInit[espcl = 4]{
        $x$     / 0.75 ,
        $f'(x)$ / 1    ,
        $f(x)$  / 1.5
    }{
                $0$                , $\ee$     , $+\infty$
    }
    \tkzTabLine{d              , + , 0         , - ,          }
    \tkzTabVar {D- / $-\infty$     , + / $\ee$ , - / $0$  }
    %
    \tkzTabVal[draw]{1}{2}{0.5} % Position entre les 1re et 2e valeurs au milieu.
                    {$1$}{$\frac{1}{\ee}$}
    \tkzTabVal[draw]{2}{3}{0.5} % Position entre les 2e et 3e valeurs au milieu.
                    {$\ee^2$}{$1$}
\end{tikzpicture}
\end{latexex-flat}


% ---------------------- %


\newparaexample{Signe à partir des variations (un peu de pédagogie...)}

Il est assez facile de produire des choses très utiles pédagogiquement comme ce qui suit en se salissant un peu les mains avec du code TikZ.

\begin{center}
\begin{tikzpicture}
    \tkzTabInit[espcl = 4]{
        $x$    / 0.75 ,
        $f(x)$ / 1.5
    }{
               $-\infty$ , $-1$     , $2$ , $+\infty$
    }
    \tkzTabVar{- /       , + / $-9$ , - / , + /      }
    %
    \tkzTabVal[draw]{3}{4}{0.5} % Position entre les 3e et 4e valeurs au milieu.
              {$3$}{$0$}
\end{tikzpicture}
\end{center}

On déduit du tableau précédent le signe de la fonction $f$.

\begin{center}
\begin{tikzpicture}
    \tkzTabInit[
        espcl = 4,
%        help       % <-- Pour voir les noms des noeuds.
    ]{
        $x$    / 0.75 ,
        $f(x)$ / 1
    }{
        $-\infty$ , , , $+\infty$ % ATTENTION ! Ne pas oublier de virgules.
    }
    %
    \node at ($(N31)!0.5!(N40)$){$3$};
    \node at ($(M31)!0.5!(M32)$){$0$};
    \node at ($(M31)!0.5!(N42)$){$+$};
    \node at ($(N11)!0.5!(M32)$){$-$};
\end{tikzpicture}
\end{center}


Voici le code du 1\ier{} tableau où le placement de la valeur $3$ au milieu entre $2$ et $+\infty$ va nous simplifier le travail pour le 2\ieme{} tableau.

\medskip

\begin{latexex-alone}
\begin{tikzpicture}
    \tkzTabInit[espcl = 4]{
        $x$    / 0.75 ,
        $f(x)$ / 1.5
    }{
               $-\infty$ , $-1$     , $2$ , $+\infty$
    }
    \tkzTabVar{- /       , + / $-9$ , - / , + /      }
    %
    \tkzTabVal[draw]{3}{4}{0.5} % Position entre les 3e et 4e valeurs au milieu.
              {$3$}{$0$}
\end{tikzpicture}
\end{latexex-alone}


Pour produire le code du 2\ieme{} tableau il a fallu utiliser au préalable ce qui suit en activant l'option \verb@help@ qui demande à \verb@tkz-tab@ d'afficher les noms de noeuds au sens TikZ qui ont été créés.
Ceci permet alors d'utiliser ces noeuds pour des dessins TikZ faits maison
\footnote{
    C'est grâce à ce mécanisme que \prefix{tnsana} peut proposer des outils explicatifs des tableaux de signe : voir la section suivante.
}.
	
\medskip

\begin{latexex-flat}
\begin{tikzpicture}
    \tkzTabInit[
        espcl = 4,
        help       % <-- Pour voir les noms des noeuds.
    ]{
        $x$    / 0.75 ,
        $f(x)$ / 1
    }{
       $-\infty$ , $-1$ , $2$ , $+\infty$
    }
\end{tikzpicture}
\end{latexex-flat}


Maintenant que ls noms des noeuds sont connus, il devient facile de produire le code ci-après.
Bien noter l'usage de valeurs utiles \og vides \fg{} de $x$ ainsi que les mystiques \verb@\node at ($(A)!0.5!(B)$)@ permettant de placer un noeud au milieu entre les deux noeuds \verb@A@ et \verb@B@. 

\medskip

\begin{latexex-alone}
\begin{tikzpicture}
    \tkzTabInit[
        espcl = 4,
%        help       % <-- Pour voir les noms des noeuds.
    ]{
        $x$    / 0.75 ,
        $f(x)$ / 1
    }{
        $-\infty$ , , , $+\infty$ % ATTENTION ! Ne pas oublier de virgules.
    }
    %
    \node at ($(N31)!0.5!(N40)$){$3$};
    \node at ($(M31)!0.5!(M32)$){$0$};
    \node at ($(M31)!0.5!(N42)$){$+$};
    \node at ($(N11)!0.5!(M32)$){$-$};
\end{tikzpicture}
\end{latexex-alone}


% ---------------------- %


\newparaexample{Convexité et concavité symbolisées dans les variations}

Voici un autre exemple s'utilisant la machinerie TikZ afin d'indiquer dans les variations la convexité et la concavité via des flèches incurvées \emph{(cette convention est proposée dans la sous-section \emph{\og Exemple utilisant l'option \macro{help} \fg} de la section \emph{\og Gallerie \fg}  de la documentation de \prefix{tkz-tab})}.

\begin{center}
\begin{tikzpicture}
    \tkzTabInit[
        espcl = 3,
%        help       % <-- Pour voir les noms des noeuds.
    ]{
        $x$   / 0.75 ,
        $f''$ / 1    ,
        $f'$  / 2    ,
        $f$   / 3
    }{
                $-\infty$          , $0$          , $+\infty$
    }
    \tkzTabLine{               , - , z        , + ,              }
    \tkzTabVar {+ / $+\infty$      , - / $-2$     , + / $+\infty$}

    \tkzTabVal[draw]{1}{2}{.5}{$x_1$}{$0$}
    \tkzTabVal[draw]{2}{3}{.5}{$x_2$}{$0$}

    \begin{scope}[->, > = triangle 60]
        \coordinate (Middle1) at ($(N13)!0.5!(N14)$);
        \coordinate (Middle2) at ($(N23)!0.5!(N24)$);
        \coordinate (Middle3) at ($(N33)!0.5!(N34)$);
        \path (N13) -- (N23) node[midway,below=6pt](N){};
        %
        \draw ([above=6pt]Middle1)
              to [bend left=45] ([left=1pt]N);
        \draw ([right=3pt]N)
              to [bend left=45] ([above=6pt]Middle2) ;
        \draw ([below right=3pt]Middle2)
              to [bend left=-45] ([above=6pt]M24) ;
        \draw ([above right=6pt]M24)
              to [bend right=40] ([below left=6pt]Middle3);
    \end{scope}
\end{tikzpicture}
\end{center}

Le code utilisé est le suivant.

\begin{latexex-alone}
\begin{tikzpicture}
    \tkzTabInit[
        espcl = 3,
%        help       % <-- Pour voir les noms des noeuds.
    ]{
        $x$   / 0.75 ,
        $f''$ / 1    ,
        $f'$  / 2    ,
        $f$   / 3
    }{
                $-\infty$          , $0$          , $+\infty$
    }
    \tkzTabLine{               , - , z        , + ,              }
    \tkzTabVar {+ / $+\infty$      , - / $-2$     , + / $+\infty$}

    \tkzTabVal[draw]{1}{2}{.5}{$x_1$}{$0$}
    \tkzTabVal[draw]{2}{3}{.5}{$x_2$}{$0$}

    \begin{scope}[->, > = triangle 60]
        \coordinate (Middle1) at ($(N13)!0.5!(N14)$);
        \coordinate (Middle2) at ($(N23)!0.5!(N24)$);
        \coordinate (Middle3) at ($(N33)!0.5!(N34)$);
        \path (N13) -- (N23) node[midway,below=6pt](N){};
        %
        \draw ([above=6pt]Middle1)
              to [bend left=45] ([left=1pt]N);
        \draw ([right=3pt]N)
              to [bend left=45] ([above=6pt]Middle2) ;
        \draw ([below right=3pt]Middle2)
              to [bend left=-45] ([above=6pt]M24) ;
        \draw ([above right=6pt]M24)
              to [bend right=40] ([below left=6pt]Middle3);
    \end{scope}
\end{tikzpicture}
\end{latexex-alone}
%% \section{Analysis}
%
%\section{Tableaux de variation et de signe}

\subsection{Décorer facilement un tableau}

\paragraph{Motivation}

Considérons le tableau suivant et imaginons que nous voulions l'expliquer à un débutant.

\begin{center}
\begin{tikzpicture}
    \tkzTabInit[
        lgt   = 3.5,  % Il faut de la place pour le dernier produit !
        espcl = 2.5   % On réduit la largeur des colonnes pour les signes.
    ]{
        $x$                             / 0.75 ,
        Signe de \\ $2 x - 3$           / 1.5  ,
        Signe de \\ $-x + 5$            / 1.5  ,
        Signe de \\ $(2 x - 3)(-x + 5)$ / 1.5
    }{
                $-\infty$     , $\frac{3}{2}$     , $5$     , $+\infty$
    }
    \tkzTabLine{          , - , z              , + , t   , + ,          }
    \tkzTabLine{          , + , t              , + , z   , - ,          }
    \tkzTabLine{          , - , z              , + , z   , - ,          }
\end{tikzpicture}
\end{center}

Deux options s'offrent à nous pour justifier comment a été rempli le tableau.

\begin{enumerate}
    \item Classiquement on résout par exemple juste les deux inéquations $2 x - 3 > 0$ et $-x + 5 > 0$ puis on complète les deux premières lignes
    \footnote{
        Notons que cette approche est un peu scandaleuse car il faudrait en toute rigueur aussi résoudre
        $2 x - 3 < 0$ , $-x + 5 < 0$ , $2 x - 3 = 0$ et $-x + 5 = 0$.
        Personne ne le fait car l'on pense aux variations d'une fonction affine. Dans ce cas pourquoi ne pas juste utiliser ce dernier argument?
        C'est ce que propose la 2\ieme{} méthode.
    }
    pour en déduire la dernière via la règle des signes d'un produit.

    \item On peut proposer une méthode moins sujette à la critique qui s'appuie sur la représentation graphique d'une fonction affine en produisant le tableau suivant.
\end{enumerate}

\begin{center}
\begin{tikzpicture}
    \tkzTabInit[
        lgt   = 3.5,  % Il faut de la place pour le dernier produit !
        espcl = 2.5   % On réduit la largeur des colonnes pour les signes.
    ]{
        $x$                             / 0.75 ,
        Signe de \\ $2 x - 3$           / 1.5  ,
        Signe de \\ $-x + 5$            / 1.5  ,
        Signe de \\ $(2 x - 3)(-x + 5)$ / 1.5
    }{
                $-\infty$     , $\frac{3}{2}$     , $5$     , $+\infty$
    }
    \tkzTabLine{          , - , z              , + , t   , + ,          }
    \tkzTabLine{          , + , t              , + , z   , - ,          }
    \tkzTabLine{          , - , z              , + , z   , - ,          }

    \comLine[gray]{0}{$\leftarrow$ Valeurs utiles de $x$}

    \graphSign        {1}{ax+b, ap}{$\frac{3}{2}$}
    \graphSign[purple]{2}{ax+b, an}{$5$}

    \comLine[gray]{3}{$\leftarrow$ Signe d'un produit.}
\end{tikzpicture}
\end{center}


Pour produire le 2\ieme{} tableau, en plus du code \verb#tkz-tab# pour le tableau de signe qui utilise les réglages optionnels \verb#lgt = 3.5# et 
\verb#espcl = 2.5# de \macro{tkzTabInit}
\footnote{
	Ceci permet d'avoir de la place dans la 1\iere{} colonne pour le dernier produit et de réduire la largeur des colonnes pour les signes.
},
il a fallu ajouter les lignes données ci-dessous où sont utilisées les macros \macro{graphSign} et \macro{comLine} proposées par \verb+tnsana+ \emph{(la syntaxe simple à suivre sera expliquée dans la section suivante)}
\footnote{
	Les lignes pour les signes doivent utiliser un coefficient minimal de \texttt{1.5} pour la hauteur afin d'éviter que la superposition des graphiques.
}.

\medskip

\begin{latexex-alone}
\begin{tikzpicture}
    % ---------------------------------------------------- %
    % -- Code tkz-tab pour les signes non reproduit ici -- %
    % ---------------------------------------------------- %
    \comLine[gray]{0}{$\leftarrow$ Valeurs utiles de $x$}

    \graphSign        {1}{ax+b, ap}{$\frac{3}{2}$}
    \graphSign[purple]{2}{ax+b, an}{$5$}

    \comLine[gray]{3}{$\leftarrow$ Signe d'un produit.}
\end{tikzpicture}
\end{latexex-alone}


\begin{remark}
	Il est aussi possible de décorer une ligne de variation comme cela sera montré dans l'exemple \ref{tnsana-grapgsign-com-two-lines} page \pageref{tnsana-grapgsign-com-two-lines}. 
\end{remark}


% ---------------------- %


\paragraph{Commenter une ligne}

L'ajout de commentaires courts se fait via la macro \macro{comLine} pour \whyprefix{com}{ment a line} soit \inenglish{commenter une ligne}
\footnote{
    L'auteur de \prefix{tnsana} n'est absolument pas un fan de la casse en bosses de chameau mais par souci de cohérence avec ce que propose \prefix{tkz-tab}, le nom \macro{comLine} a été proposé à la place de \macro{comline}.
}.
Cette macro possède un argument optionnel et deux obligatoires.

\begin{enumerate}
    \item \textbf{\emph{L'argument optionnel : choix de la couleur du texte.}}
          
          \smallskip
          
          Ci-dessus, nous avons utilisé \verb#\comLine[gray]{0}{...}# pour avoir un texte en gris.


    \medskip
    \item \textbf{\emph{Le 1\ier{} argument : numéro de ligne.}}
          
          \smallskip
          
          Par convention $0$ est le numéro de la toute 1\iere{} ligne contenant les valeurs utiles de la variable.
          \verb#\comLine{3}{...}# correspond donc à la 3\ieme{} ligne de signes ou moins intuitivement à la (3+1)\ieme{} ligne pour un humain non codeur.

    \medskip
    \item \textbf{\emph{Le 2\ieme{} argument : texte du commentaire.}}
          
          \smallskip
          
          Par défaut aucun retour à la ligne n'est possible.
          Si besoin se reporter à la  page \pageref{tnsana-grapgsign-com-two-lines} où est montré comment écrire sur plusieurs lignes \emph{(voir le tout dernier exemple de cette section)}.
\end{enumerate}


% ---------------------- %


\paragraph{Graphiques pour expliquer des signes}

Pour le moment, la macro \macro{graphSign} propose deux types de graphiques
\footnote{
    Le choix de la casse en bosses de chameau a été expliqué pour la macro \macro{comLine}.
}.
Rappelons au passage que la convention est de prendre $0$ pour numéro de la toute 1\iere{} ligne contenant les valeurs utiles de la variable.

\begin{enumerate}
    \item \textbf{\itshape Fonctions affines non constantes.}
          
          \smallskip

          Pour les fonctions du type $f(x) = a x + b$ avec $a \neq 0$, nous devons connaître le signe de $a$ et la racine $r$ de $f$.
          
          \smallskip

          Le codage est assez simple.
          Par exemple, \verb#\graphSign{2}{ax+b, an}{$5$}# indique pour la 2\ieme{} ligne d'ajouter le graphique
          d'une fonction affine, ce qu'indique le code \verb#ax+b# sans espace,
          sous la condition $a < 0$ via \prefix{an} pour \prefix{a négatif}
          et enfin avec $5$ pour racine.
          
          \smallskip

          Donc si l'on veut ajouter pour la 4\ieme{} ligne de signe le graphique de $f(x) = 3x$, on utilisera dans ce cas \verb#\graphSign{4}{ax+b, ap}{$0$}# où \prefix{ap} pour \prefix{a positif} code la condition $a > 0$.


    % ==================== %


    \medskip
    \item \textbf{\itshape Fonctions trinômiales du 2\ieme{} degré.}
          
          \smallskip

          Pour les fonctions du type $f(x) = a x^2 + b x + c$ avec $a \neq 0$, nous devons connaître le signe de $a$, celui du discriminant $\Delta = b^2 - 4ac$, ce dernier pouvant être nul, et les racines réelles éventuelles du trinôme $f$.

          \smallskip

          Voici comment coder ceci.
          Par exemple \verb#\graphSign{5}{ax2+bx+c, an, dp}{$r_1$}{$r_2$}# indique d'ajouter dans la 5\ieme{} ligne de signe le graphique 
          d'un trinôme du 2\ieme{} degré via le code \verb#ax2+bx+c# sans espace,
          sous les conditions $a < 0$ et $\Delta > 0$ via \prefix{an} et \prefix{dp},
          le trinôme ayant $r_1$ et $r_2$ pour racines réelles.

          \smallskip

          En plus de \prefix{dn} et \prefix{dp}, il y a \prefix{dz} pour \prefix{discriminant zéro}.
          Ainsi pour indiquer dans la 3\ieme{} ligne de signe la courbe relative à $f(x) = - 4 x^2$, on utilisera \verb#\graphSign{3}{ax2+bx+c, an, dz}{$0$}#.

          \smallskip

          Enfin le graphique associé au trinôme $f(x) = 7 x^2 + 3$, qui est sans racine réelle, s'obtiendra dans la 4\ieme{} ligne de signe via \verb#\graphSign{4}{ax2+bx+c, ap, dn}#.
\end{enumerate}


% ---------------------- %


\newparaexample{Un exemple avec une parabole}

Il devient très facile de proposer un tableau décoré comme le suivant.

\begin{center}
\begin{tikzpicture}
    \tkzTabInit[
        lgt   = 3   , % Il faut de la place pour le dernier produit !
        espcl = 1.5
    ]{
        $x$                        / 0.75 ,
        Signe de \\ $-x+3$         / 1.5  ,
        Signe de \\ $f(x)$         / 1.5  ,
        Signe de \\ $x^2 + 3x - 4$ / 1.5  ,
        Signe de \\ $-x^2 + x - 4$ / 1.5  ,
        Signe du \\ produit        / 1.5
    }{%
                $-\infty$     , $-4$     , $1$     , $3$ , $+\infty$
    }

    \tkzTabLine{          , + , t    , + , t   , + , z   , -        }
    \tkzTabLine{          , - , z    , + , z   , + , z   , -        }
    \tkzTabLine{          , - , z    , + , z   , - , t   , -        }
    \tkzTabLine{          , - , t    , - , t   , - , t   , -        }

    \tkzTabLine{          , - , z    , - , z   , + , z   , +        }

    \comLine[gray]{0}{\kern1.75em Schémas}
    
    \graphSign        {1}{ax+b, an}{$3$}
    \comLine[purple]  {2}{Voir Q.1-a)}
    \graphSign        {3}{ax2+bx+c, ap, dp}{$-4$}{$1$} % Deux racines réelles.
    \graphSign[purple]{4}{ax2+bx+c, an, dn}            % Aucune racine réelle.

    \comLine[gray]{5}{$\leftarrow$ Conclusion}
\end{tikzpicture}
\end{center}


En plus des deux exemples de schémas de paraboles, il faut noter dans le code supplémentaire ajouté l'utilisation de \verb#\kern1.75em# dans \verb#\comLine[gray]{0}{\kern1.75em Schémas}# afin de mettre un espace horizontal précis pour centrer à la main le texte \emph{\og Schémas \fg} \emph{(un peu sâle mais ça marche)}.

\medskip

\begin{latexex-alone}
\begin{tikzpicture}
    % ---------------------------------------------------- %
    % -- Code tkz-tab pour les signes non reproduit ici -- %
    % ---------------------------------------------------- %
    \comLine[gray]{0}{\kern1.75em Schémas}
    
    \graphSign        {1}{ax+b, an}{$3$}
    \comLine[purple]  {2}{Voir Q.1-a)}
    \graphSign        {3}{ax2+bx+c, ap, dp}{$-4$}{$1$} % Deux racines réelles.
    \graphSign[purple]{4}{ax2+bx+c, an, dn}            % Aucune racine réelle.

    \comLine[gray]{5}{$\leftarrow$ Conclusion}
\end{tikzpicture}
\end{latexex-alone}


% ---------------------- %


\newparaexample{Commenter des variations} \label{tnsana-grapgsign-com-two-lines}

Pour finir, indiquons que les outils de décoration marchent aussi pour les tableaux de variation.
Voici un exemple possible d'utilisation où le retour à la ligne a été obtenue affreusement, ou pas, via \verb#\parbox{11.5em}{Les limites sont hors programme pour cette année.}#.

\begin{center}
\begin{tikzpicture}
    \tkzTabInit[espcl = 3]{
        $x$     / 0.75 ,
        $f'(x)$ / 1    ,
        $f(x)$  / 1.5
    }{
                $0$                , $\ee$         , $+\infty$
    }
    \tkzTabLine{d              , + , 0         , - ,          }
    \tkzTabVar {D- / $-\infty$     , + / $\ee$     , - / $0$  }
    %

    \comLine{2}{%
    	\parbox{11.5em}{Les limites sont hors programme pour cette année.}%
	}
\end{tikzpicture}
\end{center}


% ---------------------- %
\section{Calcul intégral}

\subsection{Intégrales multiples}

Commençons par un point important : le package réduit les espacements entres des symboles $\int$ successifs. Voici un exemple.

\begin{latexex}
$\displaystyle
 \int \int \int 
 F(x;y;z) \dd{x} \dd{y} \dd{z}$

$\displaystyle
 \int_{a}^{b} \int_{c}^{d} \int_{e}^{f} 
 F(x;y;z) \dd{x} \dd{y} \dd{z}$
\end{latexex}


\begin{remark}
	Par défaut, \LaTeX{} affiche
	$\displaystyle
	 \stdint \stdint \stdint
	 F(x;y;z) \dd{x} \dd{y} \dd{z}$
    et
    $\displaystyle
	 \stdint_{a}^{b} \stdint_{c}^{d} \stdint_{e}^{f}
     F(x;y;z) \dd{x} \dd{y} \dd{z}$.
    Nous avons obtenu ce résultat en utilisant \macro{stdint} qui est l'opérateur proposé de façon standard par \LaTeX.
\end{remark}


% ---------------------- %


\subsection{Un opérateur d'intégration clés en main}

\newparaexample{À quoi bon ?}

Le 1\ier{} exemple qui suit semblera être une hérésie pour les habitués de \LaTeX{} mais rappelons que le but de \verb+tnsana+ est de rendre les documents facilement modifiables globalement ou localement comme le montre le 2\ieme{} exemple.

\begin{latexex}
 $\displaystyle
  \integrate{a}{b}{f(x)}{x}
= \int_{x=a}^{x=b} f(x) \dd{x}$

 $\displaystyle
  \integrate*{a}{b}{f(x)}{x}
= \integrate{a}{b}{f(x)}{x}$
\end{latexex}


\newparaexample{Le mode \texttt{displaystyle}}

La macro \macro{dintegrate*} présentée ci-dessous possède aussi une version non étoilée \macro{dintegrate}.

\begin{latexex}
 $\dintegrate*{a}{b}{f(x)}{x}
= \integrate*{a}{b}{f(x)}{x}$
\end{latexex}


% ---------------------- %


\subsection{L'opérateur crochet}

\newparaexample{}

\begin{latexex}
 $\hook{a}{b}{F(x)}{x}
= F(b) - F(a)$

 $\dintegrate*{a}{b}{f(x)}{x}
= \hook*{a}{b}{F(x)}{x}$

\end{latexex}


\begin{remark}
	Il faut savoir que \macro{hook} signifie \inenglish{crochet} mais la bonne traduction du terme mathématique est en fait \emph{\og square bracket \fg}. Ceci étant dit l'auteur de \verb+tnsana+ trouve plus efficace d'utiliser \macro{hook} comme nom de macro.
\end{remark}


% ---------------------- %


\newparaexample{Des crochets non extensibles}

Dans l'exemple suivant, on utilise l'option \prefix{sb} pour \whyprefix{s}{mall} \whyprefix{b}{rackets} soit \inenglish{petits crochets}. Les options sont disponibles à la fois pour \macro{hook} et \macro{hook*}.


\begin{latexex}
 $\hook*{a}{b}%
        {\dfrac{x - 1}{5 + x^2}}{x}
= \hook*[sb]%
        {a}{b}%
        {\dfrac{x - 1}{5 + x^2}}{x}$
\end{latexex}


% ---------------------- %


\newparaexample{Un trait vertical épuré}

Via les options \prefix{r} et \prefix{sr} pour \whyprefix{s}{mall} et \whyprefix{r}{ull} soit \inenglish*{petit} et \inenglish{trait}, on obtient ce qui suit.

\begin{latexex}
 $\hook[r]  {a}{b}%
            {\dfrac{x - 1}{5 + x^2}}{x}
= \hook*[sr]{a}{b}%
            {\dfrac{x - 1}{5 + x^2}}{x}$
\end{latexex}


% ---------------------- %
\newpage

\section{Historique}

Nous ne donnons ici qu'un très bref historique récent
\footnote{
	On ne va pas au-delà de un an depuis la dernière version.
}
de \verb+tnsana+ à destination de l'utilisateur principalement.
Tous les changements sont disponibles uniquement en anglais dans le dossier \verb+change-log+ : voir le code source de \verb+tnsana+ sur \verb+github+.

\begin{description}
% Changes shown - START

    \medskip
    \item[2020-07-15] Nouvelle version mineure \verb+0.2.0-beta+.
    
    \begin{itemize}[itemsep=.5em]
        \item \topic*{Symboles} les nouvelles macros \macro{symvar} et \macro{symvar*} produisent un disque plein et un carré plein permettant par exemple d'indiquer symboliquement une ou des variables.
        
        \separation
    \end{itemize}
% ------------------------ %

    \medskip
    \item[2020-07-12] Nouvelle version mineure \verb+0.1.0-beta+. 
        
    \begin{itemize}[itemsep=.5em]
        \item \topic*{Fonctions nommées} \macro{ppcm} et \macro{pgcd} ont été déplacées dans \texttt{tnsarith} disponible sur \url{https://github.com/typensee-latex/tnsarith.git}.
    \end{itemize}
    
    
    \separation
% ------------------------ %

    \medskip
    \item[2020-07-10] Première version \verb+0.0.0-beta+.

% ------------------------ %

% Changes shown - END 
\end{description}


\newpage
\section{Toutes les fiches techniques} \label{techincal-ids}

\subsection{Introduction}
\subsection{Constantes et paramètres}

\subsubsection{Constantes classiques}

\vspace{-1em}
\begin{multicols}{2}
% == Docs for contants - START == %

\foreach \k in {ggamma, ppi, ttau, ee, ii, jj, kk}{

	\IDmacro*{\k}{0}

}

% == Docs for contants - END == %
\vfill\null
\end{multicols}


% ---------------------- %



\subsubsection{Constantes latines personnelles}

\IDmacro*{param}{1}

\IDarg{} un texte utilisant l'alphabet latin.
\subsection{Une variable \og symbolique \fg{}}

\subsubsection{Symbole pour les opérateurs fonctionnels}

\IDmacro*{symvar}{0}

\IDmacro*{symvar*}{0}
\subsection{La fonction valeur absolue}

\subsubsection{Valeur absolue}

\IDmacro*{abs}{1}

\IDmacro*{abs*}{1}

\IDarg{} l'expression sur laquelle appliquer la fonction valeur absolue.
\subsection{Fonctions nommées spéciales}

\subsubsection{Fonctions nommées sans paramètre}

% List of functions without parameter - START

\foreach \k in {acos, asin, atan}{

    \IDmacro*{\k}{0}
}
                
\separation

\foreach \k in {arccosh, arcsinh, arctanh}{

    \IDmacro*{\k}{0}
}
                
\separation

\foreach \k in {acosh, asinh, atanh}{

    \IDmacro*{\k}{0}
}
                
\separation

\foreach \k in {fch, fsh, fth}{

    \IDmacro*{\k}{0}  où \quad \mwhyprefix{{f}}{{rench}}
}
                
\separation

\foreach \k in {afch, afsh, afth}{

    \IDmacro*{\k}{0}
}

% List of functions without parameter - END


% ---------------------- %



\subsubsection{Fonctions nommées avec un paramètre}

% List of functions with parameters - START

\IDmacro*{expb}{1}

\IDarg{} la base de l'exponentielle

\separation

\IDmacro*{logb}{1}

\IDarg{} la base du logarithme

% List of functions with parameters - END


% ---------------------- %
\subsection{Calcul différentiel}

\subsubsection{Calcul différentiel}

\IDmacro{dd}{1}{1}

\IDmacro{pp}{1}{1}

\IDoption{} utilisée, cette option sera mise en exposant du symbole $\pp{}$ ou $\dd{}$.

\IDarg{} la variable de différentiation à droite du symbole $\pp{}$ ou $\dd{}$.


% ---------------------- %



\subsubsection{Dérivation totale}

\IDmacro{der}{1}{3}

\IDoption{} la valeur par défaut est \verb+u+. 
\begin{enumerate}
	\item \verb+u+ : écriture usuelle avec des primes \emph{(ceci nécessite d'avoir une valeur entière naturelle connue du nombre de dérivations successives)}.

	\item \verb+e+ : écriture via un exposant entre des parenthèses.
	
	\item \verb+i+ : écriture via un indice.

	\item \verb+f+ : écriture via une fraction en mode display.

	\item \verb+sf+ : écriture via une fraction en mode non display.

	\item \verb+of+ : écriture via une fraction en mode display sous la forme d'un opérateur \emph{(la fonction est à côté de la fraction)}.

	\item \verb+osf+ : écriture via une fraction en mode non display sous la forme d'un opérateur \emph{(la fonction est à côté de la fraction)}.

	\smallskip
	\item \verb+p+ : ajout de parenthèses extensibles autour de la fonction.

	\item \verb+sp+ : ajout de parenthèses non extensibles autour de la fonction.
\end{enumerate}


\IDarg{1} la fonction à dériver.

\IDarg{2} l'ordre de dérivation.

\IDarg{3} la variable de dérivation.


\separation


\IDmacro{sder}{1}{2} où \quad \mwhyprefix{s}{imple}

\IDoption{} la valeur par défaut est \verb+u+. Les options disponibles sont \verb+u+, \verb+e+, \verb+p+ et \verb+sp+ : voir la fiche technique de \macro{sder} ci-dessus.

\IDarg{1} la fonction à dériver.

\IDarg{2} l'ordre de dérivation.


% ---------------------- %


\subsubsection{Dérivation totale -- Opérateur fonctionnel}

\IDmacro{derope}{1}{2} où \quad \mwhyprefix{ope}{rator}

\IDoption{} la valeur par défaut est \verb+f+. Les options disponibles sont \verb+f+, \verb+sf+ et \verb+i+ : voir la fiche technique de \macro{der} donnée un peu plus haut.

\IDarg{1} la fonction à dériver.

\IDarg{2} l'ordre de dérivation.
\subsubsection{Dérivation partielle}

\IDmacro{pder}{1}{2} où \quad \mwhyprefix{p}{artial}

\IDoption{} la valeur par défaut est \verb+f+. 
\begin{enumerate}
	\item \verb+f+ : écriture via une fraction en mode display.

	\item \verb+sf+ : écriture via une fraction en mode non display.

	\item \verb+of+ : écriture via une fraction en mode display sous la forme d'un opérateur \emph{(la fonction est à côté de la fraction)}.

	\item \verb+osf+ : écriture via une fraction en mode non display sous la forme d'un opérateur \emph{(la fonction est à côté de la fraction)}.

	\item \verb+i+ : écriture via un indice.

	\smallskip
	\item \verb+p+ : ajout de parenthèses extensibles autour de la fonction.

	\item \verb+sp+ : ajout de parenthèses non extensibles autour de la fonction.
\end{enumerate}


\IDarg{1} la fonction à dériver.

\IDarg{2} les variables utilisées avec leur ordre de dérivation pour la dérivation partielle en utilisant une syntaxe du type \verb+x | y^2 | ...+ qui indique de dériver suivant $x$ une fois, puis suivant $y$ trois fois... etc.

\IDarg{3} l'ordre total de dérivation.


% ---------------------- %


\subsubsection{Dérivation partielle - Opérateur fonctionnel}

\IDmacro{pderope}{1}{2} où \quad \mwhyprefix{p}{artial}
                              et \mwhyprefix{ope}{rator}

\IDoption{} la valeur par défaut est \verb+f+. Les options disponibles sont \verb+f+, \verb+sf+ et \verb+i+ : voir la fiche technique de \macro{pder} juste avant.

\IDarg{1} les variables utilisées avec leur ordre de dérivation via la syntaxe indiquée ci-dessus.

\IDarg{2} l'ordre total de dérivation.

\subsubsection{Tableaux de signes -- Commentaires et graphiques explicatifs}

\IDmacro{comLine}{1}{2}  où \quad \mwhyprefix{com}{ment}


\IDoption{} couleur au format TikZ.


\IDarg{1} le numéro de ligne où placer le commentaire, $0$ étant le 1\ier{} numéro.

\IDarg{2} le texte du commentaire.


\separation


\IDmacro{graphSign}{1}{2..4}


\IDoption{} couleur au format TikZ.


\IDarg{1} le numéro de ligne où placer le graphique, $0$ étant le 1\ier{} numéro.


\IDarg{2} le type de fonctions avec des contraintes en utilisant la virgule comme séparateur d'informations.

\begin{enumerate}
	\item \verb@ax+b@ sans espace indique une fonction affine avec un unique paramètre \verb@a@ non nul à caractériser.

	\item \verb@ax2+bx+c@ sans espace indique une fonction trinôme du 2\ieme{} degré avec une paramètre \verb@a@ non nul à caractériser ainsi que \verb@d@ pour son discriminant.

    % ==================== %

	\smallskip
	
	\item \verb@ap@ et \verb@an@ indiquent respectivement les conditions $a > 0$ et $a < 0$.

	\item \verb@dp@, \verb@dz@ et \verb@dn@ indiquent respectivement les conditions $d > 0$, $d = 0$ et $d < 0$.
\end{enumerate}


\IDarg{supplémentaire pour ax+b} la racine de $ax + b$.


\IDarg{$\!$s supplémentaires pour ax2+bx+c} si $ax^2 + bx + c$ admet une ou deux racines, on donnera toutes les racines de la plus petite à la plus grande
\footnote{
	Notant $\Delta = b^2 - 4 ac$, si $\Delta < 0$ il n'y aura pas d'argument supplémentaire, si $\Delta = 0$ il y en aura un seul et enfin si $\Delta > 0$ il faudra en donner deux, le 1\ier{} étant le plus petit.
}.
\subsection{Calcul intégral}

\subsubsection{Intégration -- Le symbole standard}

\IDmacro*{stdint}{0}


% ---------------------- %



\subsubsection{Intégration -- Fonctionnelle d'intégration}

\IDmacro*{integrate}{4}

\IDmacro*{integrate*}{4}

\extraspace

\IDmacro*{dintegrate}{4}   où \quad \mwhyprefix{d}{isplaystyle}

\IDmacro*{dintegrate*}{4}  où \quad \mwhyprefix{d}{isplaystyle}

\IDarg{1} ce qui est en bas du symbole $\int_{\symvar}$ .

\IDarg{2} ce qui est en haut du symbole $\int^{\symvar}$ .

\IDarg{3} la fonction intégrée.

\IDarg{4} suivant quoi on intègre.


% ---------------------- %



\subsubsection{Intégration -- L'opérateur crochet}

\IDmacro{hook}{1}{4}

\IDmacro{hook*}{1}{4}

\IDoption{} la valeur par défaut est \verb+b+. Voici les différentes valeurs possibles.
\begin{enumerate}
	\item \verb+b+ : des crochets extensibles sont utilisés.

	\item \verb+sb+ : des crochets non extensibles sont utilisés.

	\item \verb+r+ : un unique trait vertical extensible est utilisé à droite.

	\item \verb+sr+ : un unique trait vertical non extensible est utilisé à droite.
\end{enumerate}

\IDarg{1} ce qui est en bas du crochet fermant.

\IDarg{2} ce qui est en haut du crochet fermant.

\IDarg{3} la fonction sur laquelle effectuer le calcul.

\IDarg{4} la variable pour les calculs.
\newpage

\end{document}
