\documentclass[12pt,a4paper]{article}

\makeatletter
    \usepackage[utf8]{inputenc}
\usepackage[T1]{fontenc}
\usepackage{ucs}

\usepackage[french]{babel,varioref}

\usepackage[top=2cm, bottom=2cm, left=1.5cm, right=1.5cm]{geometry}
\usepackage{enumitem}

\usepackage{multicol}

\usepackage{makecell}

\usepackage{color}
\usepackage{hyperref}
\hypersetup{
    colorlinks,
    citecolor=black,
    filecolor=black,
    linkcolor=black,
    urlcolor=black
}

\usepackage{amsthm}

\usepackage{tcolorbox}
\tcbuselibrary{listingsutf8}

\usepackage{ifplatform}

\usepackage{ifthen}

\usepackage{cbdevtool}



% MISC

\newtcblisting{latexex}{%
	sharp corners,%
	left=1mm, right=1mm,%
	bottom=1mm, top=1mm,%
	colupper=red!75!blue,%
	listing side text
}

\newtcblisting{latexex-flat}{%
	sharp corners,%
	left=1mm, right=1mm,%
	bottom=1mm, top=1mm,%
	colupper=red!75!blue,%
}

\newtcblisting{latexex-alone}{%
	sharp corners,%
	left=1mm, right=1mm,%
	bottom=1mm, top=1mm,%
	colupper=red!75!blue,%
	listing only
}


\newcommand\env[1]{\texttt{#1}}
\newcommand\macro[1]{\env{\textbackslash{}#1}}



\setlength{\parindent}{0cm}
\setlist{noitemsep}

\theoremstyle{definition}
\newtheorem*{remark}{Remarque}

\usepackage[raggedright]{titlesec}

\titleformat{\paragraph}[hang]{\normalfont\normalsize\bfseries}{\theparagraph}{1em}{}
\titlespacing*{\paragraph}{0pt}{3.25ex plus 1ex minus .2ex}{0.5em}


\newcommand\separation{
	\medskip
	\hfill\rule{0.5\textwidth}{0.75pt}\hfill
	\medskip
}


\newcommand\extraspace{
	\vspace{0.25em}
}


\newcommand\whyprefix[2]{%
	\textbf{\prefix{#1}}-#2%
}

\newcommand\mwhyprefix[2]{%
	\texttt{#1 = #1-#2}%
}

\newcommand\prefix[1]{%
	\texttt{#1}%
}


\newcommand\inenglish{\@ifstar{\@inenglish@star}{\@inenglish@no@star}}

\newcommand\@inenglish@star[1]{%
	\emph{\og #1 \fg}%
}

\newcommand\@inenglish@no@star[1]{%
	\@inenglish@star{#1} en anglais%
}


\newcommand\ascii{\texttt{ASCII}}


% Example
\newcounter{paraexample}[subsubsection]

\newcommand\@newexample@abstract[2]{%
	\paragraph{%
		#1%
		\if\relax\detokenize{#2}\relax\else {} -- #2\fi%
	}%
}



\newcommand\newparaexample{\@ifstar{\@newparaexample@star}{\@newparaexample@no@star}}

\newcommand\@newparaexample@no@star[1]{%
	\refstepcounter{paraexample}%
	\@newexample@abstract{Exemple \theparaexample}{#1}%
}

\newcommand\@newparaexample@star[1]{%
	\@newexample@abstract{Exemple}{#1}%
}


% Change log
\newcommand\topic{\@ifstar{\@topic@star}{\@topic@no@star}}

\newcommand\@topic@no@star[1]{%
	\textbf{\textsc{#1}.}%
}

\newcommand\@topic@star[1]{%
	\textbf{\textsc{#1} :}%
}


\makeatother


\begin{document}

\newpage

\section{Historique}

Nous ne donnons ici qu'un très bref historique récent
\footnote{
	On ne va pas au-delà de un an depuis la dernière version.
}
de \verb+tnsana+ à destination de l'utilisateur principalement.
Tous les changements sont disponibles uniquement en anglais dans le dossier \verb+change-log+ : voir le code source de \verb+tnsana+ sur \verb+github+.

\begin{description}
% Changes shown - START

    \medskip
    \item[2020-07-30] Nouvelle version mineure \verb+0.6.0-beta+.
    
    \begin{itemize}[itemsep=.5em]
        \item \topic{Définition explicite d'une fonction}
        \begin{itemize}[itemsep=.5em]
            \item Une nouvelle macro \macro{txfuncdef} produit une version textuelle courte.
    
            \item Omission possible des ensembles, via des arguments vides, quand on utilise \macro{funcdef[h]} ou \macro{txfuncdef}.
        \end{itemize}
    
    % ------------------------ %
    
        \item \topic*{Fonctions avec un paramètre}
              les macros \macro{expb} et \macro{logb} ont été remplacées par \macro{exp} et \macro{log} qui ont un argument optionnel pour indiquer éventuellement une base.
    
    % ------------------------ %
    
        \item \topic*{Dérivation partielle}
              \macro{pder[ei]} fonctionne maintenant aussi avec des variables indexées.
              
    \end{itemize}
    
    \separation
% ------------------------ %

    \medskip
    \item[2020-07-22] Nouvelle version mineure \verb+0.5.0-beta+.
    
    \begin{itemize}[itemsep=.5em]
        \item \topic*{Définition explicite d'une fonction}
              ajout de \macro{funcdef}.
    \end{itemize}
    
    \separation
% ------------------------ %

    \medskip
    \item[2020-07-21] Nouvelle version mineure \verb+0.4.0-beta+.
    
    \begin{itemize}[itemsep=.5em]
        \item \topic*{Limite}
              ajout de \macro{limit} pour l'écriture de limites de fonctions à une seule variable.
    
    % ------------------------ %
    
        \item \topic*{Dérivation}
              par souci de cohérence, il faudra taper \verb#\der{f}{x}{n}# au lieu de l'ancien \verb#\der{f}{n}{x}#.
    
    % ------------------------ %
    
        \item \topic*{Intégration}
              par souci de cohérence, il faudra taper \verb#\integrate{f}{x}{a}{b}# au lieu de l'ancien \verb#\integrate{a}{b}{f}{x}#.
              Il en va de même pour \macro{hook}.
    \end{itemize}
    
    \separation

% ------------------------ %

    \medskip
    \item[2020-07-17] Nouvelle version mineure \verb+0.3.0-beta+.
    
    \begin{itemize}[itemsep=.5em]
        \item \topic{Dérivation}
        \begin{itemize}[itemsep=.5em]
            \item Dérivation pointée à la physicienne via \verb+d+ et \verb+bd+ deux nouvelles options de \macro{der}.
    
            \item La dérivation partielle indexée du type $u_{xxy}$ à la physicienne via \verb+ei+ une nouvelle option de \macro{pder}.
        \end{itemize}
    
    % ------------------------ %
    
        \item \topic{Tableaux de signe et de variation}
        \begin{itemize}[itemsep=.5em]
            \item Ajout de \macro{backLine} pour changer la couleur de fond d'une ou plusieurs lignes.
    
    
            \item \macro{graphSign} propose des fonctions de référence \emph{(sans paramètre)}.
        \end{itemize}
    \end{itemize}
    
    \separation

% ------------------------ %

    \medskip
    \item[2020-07-15] Nouvelle version mineure \verb+0.2.0-beta+.
    
    \begin{itemize}[itemsep=.5em]
        \item \topic*{Symboles} les nouvelles macros \macro{symvar} et \macro{symvar*} produisent un disque plein et un carré plein permettant par exemple d'indiquer symboliquement une ou des variables.
    \end{itemize}
    
    \separation
% ------------------------ %

    \medskip
    \item[2020-07-12] Nouvelle version mineure \verb+0.1.0-beta+. 
        
    \begin{itemize}[itemsep=.5em]
        \item \topic*{Fonctions nommées} \macro{ppcm} et \macro{pgcd} ont été déplacées dans \texttt{tnsarith} disponible sur \url{https://github.com/typensee-latex/tnsarith.git}.
    \end{itemize}
    
    \separation

% ------------------------ %

    \medskip
    \item[2020-07-10] Première version \verb+0.0.0-beta+.

% ------------------------ %

% Changes shown - END 
\end{description}

\end{document}
