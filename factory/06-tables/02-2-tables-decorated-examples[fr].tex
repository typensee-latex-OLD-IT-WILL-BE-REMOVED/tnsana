\documentclass[12pt,a4paper]{article}

\makeatletter
    \usepackage[utf8]{inputenc}
\usepackage[T1]{fontenc}
\usepackage{ucs}

\usepackage[french]{babel,varioref}

\usepackage[top=2cm, bottom=2cm, left=1.5cm, right=1.5cm]{geometry}
\usepackage{enumitem}

\usepackage{multicol}

\usepackage{makecell}

\usepackage{color}
\usepackage{hyperref}
\hypersetup{
    colorlinks,
    citecolor=black,
    filecolor=black,
    linkcolor=black,
    urlcolor=black
}

\usepackage{amsthm}

\usepackage{tcolorbox}
\tcbuselibrary{listingsutf8}

\usepackage{ifplatform}

\usepackage{ifthen}

\usepackage{cbdevtool}



% MISC

\newtcblisting{latexex}{%
	sharp corners,%
	left=1mm, right=1mm,%
	bottom=1mm, top=1mm,%
	colupper=red!75!blue,%
	listing side text
}

\newtcblisting{latexex-flat}{%
	sharp corners,%
	left=1mm, right=1mm,%
	bottom=1mm, top=1mm,%
	colupper=red!75!blue,%
}

\newtcblisting{latexex-alone}{%
	sharp corners,%
	left=1mm, right=1mm,%
	bottom=1mm, top=1mm,%
	colupper=red!75!blue,%
	listing only
}


\newcommand\env[1]{\texttt{#1}}
\newcommand\macro[1]{\env{\textbackslash{}#1}}



\setlength{\parindent}{0cm}
\setlist{noitemsep}

\theoremstyle{definition}
\newtheorem*{remark}{Remarque}

\usepackage[raggedright]{titlesec}

\titleformat{\paragraph}[hang]{\normalfont\normalsize\bfseries}{\theparagraph}{1em}{}
\titlespacing*{\paragraph}{0pt}{3.25ex plus 1ex minus .2ex}{0.5em}


\newcommand\separation{
	\medskip
	\hfill\rule{0.5\textwidth}{0.75pt}\hfill
	\medskip
}


\newcommand\extraspace{
	\vspace{0.25em}
}


\newcommand\whyprefix[2]{%
	\textbf{\prefix{#1}}-#2%
}

\newcommand\mwhyprefix[2]{%
	\texttt{#1 = #1-#2}%
}

\newcommand\prefix[1]{%
	\texttt{#1}%
}


\newcommand\inenglish{\@ifstar{\@inenglish@star}{\@inenglish@no@star}}

\newcommand\@inenglish@star[1]{%
	\emph{\og #1 \fg}%
}

\newcommand\@inenglish@no@star[1]{%
	\@inenglish@star{#1} en anglais%
}


\newcommand\ascii{\texttt{ASCII}}


% Example
\newcounter{paraexample}[subsubsection]

\newcommand\@newexample@abstract[2]{%
	\paragraph{%
		#1%
		\if\relax\detokenize{#2}\relax\else {} -- #2\fi%
	}%
}



\newcommand\newparaexample{\@ifstar{\@newparaexample@star}{\@newparaexample@no@star}}

\newcommand\@newparaexample@no@star[1]{%
	\refstepcounter{paraexample}%
	\@newexample@abstract{Exemple \theparaexample}{#1}%
}

\newcommand\@newparaexample@star[1]{%
	\@newexample@abstract{Exemple}{#1}%
}


% Change log
\newcommand\topic{\@ifstar{\@topic@star}{\@topic@no@star}}

\newcommand\@topic@no@star[1]{%
	\textbf{\textsc{#1}.}%
}

\newcommand\@topic@star[1]{%
	\textbf{\textsc{#1} :}%
}


    % == PACKAGES USED == %

\RequirePackage[Symbolsmallscale]{upgreek}
\RequirePackage{xstring}


% == DEFINITIONS == %

% Constants - START

% User's constants

\newcommand\param[1]{%
    \IfStrEqCase{#1}{%
        {gamma}{\upgamma}%
        {pi}{\uppi}%
        {tau}{\uptau}%
    }[\text{\textbf{#1}}]
}

% Classical constants
    
\newcommand\ggamma{\param{gamma}}
\newcommand\ppi{\param{pi}}
\newcommand\ttau{\param{tau}}
\newcommand\ee{\param{e}}
\newcommand\ii{\param{i}}
\newcommand\jj{\param{j}}
\newcommand\kk{\param{k}}

% Constants - END

    % == PACKAGES USED == %

\RequirePackage{mathtools}


% == DEFINITIONS == %

% Source :
%    * https://tex.stackexchange.com/a/43009/6880
%
\DeclarePairedDelimiter\abs{\lvert}{\rvert}%

\let\@old@abs\abs
\def\abs{\@ifstar{\@old@abs}{\@old@abs*}}

	\usepackage{amsmath}

    \usepackage{01-tables}
\makeatother


\begin{document}

%% \section{Analysis}
%
%\section{Tableaux de variation et de signe}

\subsubsection{Quelques exemples}  \label{tnsana-graphsign-examples}

\newparaexample{Avec une parabole}

Il devient très facile de proposer un tableau décoré comme le suivant.

\begin{center}
	\input{tikz/deco-parabola.tkz}
\end{center}


En plus des deux exemples de schémas de paraboles, il faut noter dans le code supplémentaire ajouté l'utilisation de \verb#\kern1.75em# dans \verb#\comLine[gray]{0}{\kern1.75em Schémas}# afin de mettre un espace horizontal précis pour centrer à la main le texte \emph{\og Schémas \fg} \emph{(un peu sâle mais ça marche)}.

\medskip

\inputlatexexalone{tikz/deco-parabola-short.tkz}


% ---------------------- %


\newparaexample{Avec des fonctions sans paramètre}

Voici un 1\ier{} tableau avec certaines des fonctions sans paramètre.

\begin{center}
	\input{tikz/deco-ref-1.tkz}
\end{center}

Le code correspondant est le suivant.

\medskip

\inputlatexexalone{tikz/deco-ref-1-short.tkz}

\medskip

Voici un 2\ieme{} tableau avec les fonctions sans paramètre manquantes ci-dessus.

\begin{center}
	\input{tikz/deco-ref-2.tkz}
\end{center}

Le code correspondant est le suivant.

\medskip

\inputlatexexalone{tikz/deco-ref-2-short.tkz}


% ---------------------- %


\newparaexample{Commenter des variations} \label{tnsana-graphsign-com-two-lines}

Pour finir, indiquons que les outils de décoration marchent aussi pour les tableaux de variation.
Voici un exemple possible d'utilisation où les retours à la ligne ont été obtenus affreusement, ou pas, via \verb#\parbox{11.5em}{...}#.

\begin{center}
	\input{tikz/deco-var.tkz}
\end{center}

\end{document}
