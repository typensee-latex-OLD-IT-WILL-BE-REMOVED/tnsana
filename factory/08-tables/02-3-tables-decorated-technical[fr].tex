\documentclass[12pt,a4paper]{article}

\makeatletter
    \usepackage[utf8]{inputenc}
\usepackage[T1]{fontenc}
\usepackage{ucs}

\usepackage[french]{babel,varioref}

\usepackage[top=2cm, bottom=2cm, left=1.5cm, right=1.5cm]{geometry}
\usepackage{enumitem}

\usepackage{multicol}

\usepackage{makecell}

\usepackage{color}
\usepackage{hyperref}
\hypersetup{
    colorlinks,
    citecolor=black,
    filecolor=black,
    linkcolor=black,
    urlcolor=black
}

\usepackage{amsthm}

\usepackage{tcolorbox}
\tcbuselibrary{listingsutf8}

\usepackage{ifplatform}

\usepackage{ifthen}

\usepackage{cbdevtool}



% MISC

\newtcblisting{latexex}{%
	sharp corners,%
	left=1mm, right=1mm,%
	bottom=1mm, top=1mm,%
	colupper=red!75!blue,%
	listing side text
}

\newtcblisting{latexex-flat}{%
	sharp corners,%
	left=1mm, right=1mm,%
	bottom=1mm, top=1mm,%
	colupper=red!75!blue,%
}

\newtcblisting{latexex-alone}{%
	sharp corners,%
	left=1mm, right=1mm,%
	bottom=1mm, top=1mm,%
	colupper=red!75!blue,%
	listing only
}


\newcommand\env[1]{\texttt{#1}}
\newcommand\macro[1]{\env{\textbackslash{}#1}}



\setlength{\parindent}{0cm}
\setlist{noitemsep}

\theoremstyle{definition}
\newtheorem*{remark}{Remarque}

\usepackage[raggedright]{titlesec}

\titleformat{\paragraph}[hang]{\normalfont\normalsize\bfseries}{\theparagraph}{1em}{}
\titlespacing*{\paragraph}{0pt}{3.25ex plus 1ex minus .2ex}{0.5em}


\newcommand\separation{
	\medskip
	\hfill\rule{0.5\textwidth}{0.75pt}\hfill
	\medskip
}


\newcommand\extraspace{
	\vspace{0.25em}
}


\newcommand\whyprefix[2]{%
	\textbf{\prefix{#1}}-#2%
}

\newcommand\mwhyprefix[2]{%
	\texttt{#1 = #1-#2}%
}

\newcommand\prefix[1]{%
	\texttt{#1}%
}


\newcommand\inenglish{\@ifstar{\@inenglish@star}{\@inenglish@no@star}}

\newcommand\@inenglish@star[1]{%
	\emph{\og #1 \fg}%
}

\newcommand\@inenglish@no@star[1]{%
	\@inenglish@star{#1} en anglais%
}


\newcommand\ascii{\texttt{ASCII}}


% Example
\newcounter{paraexample}[subsubsection]

\newcommand\@newexample@abstract[2]{%
	\paragraph{%
		#1%
		\if\relax\detokenize{#2}\relax\else {} -- #2\fi%
	}%
}



\newcommand\newparaexample{\@ifstar{\@newparaexample@star}{\@newparaexample@no@star}}

\newcommand\@newparaexample@no@star[1]{%
	\refstepcounter{paraexample}%
	\@newexample@abstract{Exemple \theparaexample}{#1}%
}

\newcommand\@newparaexample@star[1]{%
	\@newexample@abstract{Exemple}{#1}%
}


% Change log
\newcommand\topic{\@ifstar{\@topic@star}{\@topic@no@star}}

\newcommand\@topic@no@star[1]{%
	\textbf{\textsc{#1}.}%
}

\newcommand\@topic@star[1]{%
	\textbf{\textsc{#1} :}%
}


    % == PACKAGES USED == %

\RequirePackage[Symbolsmallscale]{upgreek}
\RequirePackage{xstring}


% == DEFINITIONS == %

% Constants - START

% User's constants

\newcommand\param[1]{%
    \IfStrEqCase{#1}{%
        {gamma}{\upgamma}%
        {pi}{\uppi}%
        {tau}{\uptau}%
    }[\text{\textbf{#1}}]
}

% Classical constants
    
\newcommand\ggamma{\param{gamma}}
\newcommand\ppi{\param{pi}}
\newcommand\ttau{\param{tau}}
\newcommand\ee{\param{e}}
\newcommand\ii{\param{i}}
\newcommand\jj{\param{j}}
\newcommand\kk{\param{k}}

% Constants - END

    % == PACKAGES USED == %

\RequirePackage{mathtools}


% == DEFINITIONS == %

% Source :
%    * https://tex.stackexchange.com/a/43009/6880
%
\DeclarePairedDelimiter\abs{\lvert}{\rvert}%

\let\@old@abs\abs
\def\abs{\@ifstar{\@old@abs}{\@old@abs*}}

	\usepackage{amsmath}

    \usepackage{01-tables}
\makeatother


\begin{document}

%% \section{Analysis}
%
%\section{Tableaux de variation et de signe}

\section{Fiche technique}

\subsection{Coloriser le fond}

\IDmacro{backLine}{1}{1} \hfill \mwhyprefix{back}{ground}


\IDoption{} couleur au format TikZ.
            La valeur par défaut est \verb+gray!30+.


\IDarg{1} les numéros de ligne séparés par des virgules, $0$ étant le 1\ier{} numéro.


% ---------------------- %


\subsection{Commentaires}

\IDmacro{comLine}{1}{2} \hfill \mwhyprefix{com}{ment}


\IDoption{} couleur au format TikZ.
            La valeur par défaut est \verb+blue+.


\IDarg{1} le numéro de ligne, $0$ étant le 1\ier{} numéro.

\IDarg{2} le texte du commentaire.


% ---------------------- %


\subsection{Graphiques explicatifs}

\IDmacro{graphSign}{1}{4}


\IDoption{} couleur au format TikZ.
            La valeur par défaut est \verb+blue+.


\IDarg{1} le numéro de ligne, $0$ étant le 1\ier{} numéro.


\IDarg{2} le type de fonctions avec des contraintes éventuelles en utilisant la virgule comme séparateur d'informations.

\begin{enumerate}
	\item \verb@x2  @ sans espace indique $f(x) = x^2$.

	\item \verb@srqt@ sans espace indique $f(x) = \sqrt{x}$.

	\item \verb@1/x @ sans espace indique $f(x) = \frac{1}{x}$.

	\item \verb@abs @ sans espace indique $f(x) = \abs{x}$.

	\item \verb@exp @ sans espace indique $f(x) = \exp x$.

	\item \verb@ln  @ sans espace indique $f(x) = \ln x$.

    % ==================== %

	\medskip
	
	\item \verb@ax+b    @ sans espace indique $f(x) = ax + b$ avec $a \neq 0$ à caractériser.

	\item \verb@ax2+bx+c@ sans espace indique $f(x) = ax^2 + bx + c$ avec $a \neq 0$ et le discriminant $d$ à caractériser.

    % ==================== %

	\medskip
	
	\item \verb@ap@ et \verb@an@ indiquent respectivement les conditions $a > 0$ et $a < 0$.

	\item \verb@dp@, \verb@dz@ et \verb@dn@ indiquent respectivement les conditions $d > 0$, $d = 0$ et $d < 0$.
\end{enumerate}


\IDarg{3 supplémentaire pour ax+b} la racine réelle de $ax + b$.


\IDarg{$\!$s supplémentaires éventuels pour ax2+bx+c} si $ax^2 + bx + c$ admet une ou deux racines réelles, on donnera toutes les racines de la plus petite à la plus grande
\footnote{
	Notant $\Delta = b^2 - 4 ac$, si $\Delta < 0$ il n'y aura pas d'argument supplémentaire, si $\Delta = 0$ il y en aura un seul et enfin si $\Delta > 0$ il faudra en donner deux, le 1\ier{} étant le plus petit.
}.

\end{document}
