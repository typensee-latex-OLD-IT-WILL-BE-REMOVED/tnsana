\documentclass[12pt,a4paper]{article}


\makeatletter
    \usepackage[utf8]{inputenc}
\usepackage[T1]{fontenc}
\usepackage{ucs}

\usepackage[french]{babel,varioref}

\usepackage[top=2cm, bottom=2cm, left=1.5cm, right=1.5cm]{geometry}
\usepackage{enumitem}

\usepackage{multicol}

\usepackage{makecell}

\usepackage{color}
\usepackage{hyperref}
\hypersetup{
    colorlinks,
    citecolor=black,
    filecolor=black,
    linkcolor=black,
    urlcolor=black
}

\usepackage{amsthm}

\usepackage{tcolorbox}
\tcbuselibrary{listingsutf8}

\usepackage{ifplatform}

\usepackage{ifthen}

\usepackage{cbdevtool}



% MISC

\newtcblisting{latexex}{%
	sharp corners,%
	left=1mm, right=1mm,%
	bottom=1mm, top=1mm,%
	colupper=red!75!blue,%
	listing side text
}

\newtcblisting{latexex-flat}{%
	sharp corners,%
	left=1mm, right=1mm,%
	bottom=1mm, top=1mm,%
	colupper=red!75!blue,%
}

\newtcblisting{latexex-alone}{%
	sharp corners,%
	left=1mm, right=1mm,%
	bottom=1mm, top=1mm,%
	colupper=red!75!blue,%
	listing only
}


\newcommand\env[1]{\texttt{#1}}
\newcommand\macro[1]{\env{\textbackslash{}#1}}



\setlength{\parindent}{0cm}
\setlist{noitemsep}

\theoremstyle{definition}
\newtheorem*{remark}{Remarque}

\usepackage[raggedright]{titlesec}

\titleformat{\paragraph}[hang]{\normalfont\normalsize\bfseries}{\theparagraph}{1em}{}
\titlespacing*{\paragraph}{0pt}{3.25ex plus 1ex minus .2ex}{0.5em}


\newcommand\separation{
	\medskip
	\hfill\rule{0.5\textwidth}{0.75pt}\hfill
	\medskip
}


\newcommand\extraspace{
	\vspace{0.25em}
}


\newcommand\whyprefix[2]{%
	\textbf{\prefix{#1}}-#2%
}

\newcommand\mwhyprefix[2]{%
	\texttt{#1 = #1-#2}%
}

\newcommand\prefix[1]{%
	\texttt{#1}%
}


\newcommand\inenglish{\@ifstar{\@inenglish@star}{\@inenglish@no@star}}

\newcommand\@inenglish@star[1]{%
	\emph{\og #1 \fg}%
}

\newcommand\@inenglish@no@star[1]{%
	\@inenglish@star{#1} en anglais%
}


\newcommand\ascii{\texttt{ASCII}}


% Example
\newcounter{paraexample}[subsubsection]

\newcommand\@newexample@abstract[2]{%
	\paragraph{%
		#1%
		\if\relax\detokenize{#2}\relax\else {} -- #2\fi%
	}%
}



\newcommand\newparaexample{\@ifstar{\@newparaexample@star}{\@newparaexample@no@star}}

\newcommand\@newparaexample@no@star[1]{%
	\refstepcounter{paraexample}%
	\@newexample@abstract{Exemple \theparaexample}{#1}%
}

\newcommand\@newparaexample@star[1]{%
	\@newexample@abstract{Exemple}{#1}%
}


% Change log
\newcommand\topic{\@ifstar{\@topic@star}{\@topic@no@star}}

\newcommand\@topic@no@star[1]{%
	\textbf{\textsc{#1}.}%
}

\newcommand\@topic@star[1]{%
	\textbf{\textsc{#1} :}%
}



    \usepackage{01-total-diff-calculus}


   
\begin{document}

\section{Calcul différentiel}

\subsection{\texorpdfstring{Les opérateurs $\pp{}$ et $\dd{}$}%
                           {Les opérateurs "d rond" et "d droit"}}

Voici deux opérateurs utiles aussi bien pour du calcul différentiel que du calcul intégral. 

\begin{latexex}
$\dd{t} = \dd[1]{t}$ ou $\dd[n]{x}$

$\pp{t} = \pp[1]{t}$ ou $\pp[n]{x}$
\end{latexex}


% ---------------------- %


\subsection{Fiches techniques}

\IDmacro{dd}{1}{1}

\IDmacro{pp}{1}{1}

\IDoption{} utilisée, cette option sera mise en exposant du symbole $\pp{}$ ou $\dd{}$.

\IDarg{} la variable de différentiation à droite du symbole $\pp{}$ ou $\dd{}$.


% ---------------------- %


\subsection{Dérivations totales d'une fonction -- Version longue avec une variable}

\newparaexample{Différentes écritures possibles}

La macro \macro{der} est stricte du point de vue sémantique car on doit lui fournir la fonction, l'ordre de dérivation et la variable de dérivation
\emph{(voir la section \ref{tnsana-short-der} qui présente la macro \macro{sder} permettant une rédaction efficace pour obtenir $\sder[e]{f}{1}$ ou $\sder{f}{1}$)}.
Voici plusieurs mises en forme faciles à taper via l'option de \macro{der}.
Attention bien entendu à n'utiliser l'option par défaut \prefix{u} ou l'option \prefix{d} qu'avec un ordre de dérivation de valeur naturelle connue !

\begin{latexex}
 $\der   {f}{x}{3}
= \der[e]{f}{x}{3}
= \der[d]{f}{x}{3}$

 $\der[i] {u}{x}{k}
= \der[f] {u}{x}{k}
= \der[sf]{u}{x}{k}$
\end{latexex}


On peut aussi ajouter autour de la fonction des parenthèses extensibles ou non sauf même si cela n'est pas utile pour le mode \prefix{d} \emph{(voir juste après le mode \prefix{bd})}.
Ci-dessous on montre au passage une écriture du type \emph{\og opérateur fonctionnel \fg} : voir la section \ref{tnsana-ope-total-der} page \pageref{tnsana-ope-total-der} à ce sujet.

\begin{latexex}
 $\der[osf,sp]{\frac{1}{2}  uv}{x}{k}
= \der[of,p]  {\dfrac{1}{2} uv}{x}{k}$
\end{latexex}


Avec l'option \prefix{d} les parenthèses seules sont sans utilité car on peut obtenir des choses non souhaitées comme $\der[d, p]{u + v}{x}{2}$.
À la place on utilisera l'option \prefix{bd} où le \prefix{b} est pour \whyprefix{b}{racket} soit \inenglish{crochet}. Notez au passage que \prefix{p} et \prefix{sp} restent utilisables.

\begin{latexex}
 $\der[bd]   {\frac{1}{2}  uv}{x}{2}
= \der[bd,p] {\frac{1}{2}  uv}{x}{2}
= \der[bd,sp]{\dfrac{1}{2} uv}{x}{2}$
\end{latexex}


\begin{remark}
	Expliquons les valeurs des options.
	\begin{enumerate}
		\item \prefix{u}, la valeur par défaut, est pour \whyprefix{u}{suel} soit l'écriture avec les primes. Cette option ne marchera pas avec un nombre symbolique de dérivations. 

		\item \prefix{e} est pour \whyprefix{e}{xposant}.

		\item \prefix{i} est pour \whyprefix{i}{ndice}.

		\item \prefix{d} est pour \whyprefix{d}{ot} soit \inenglish{point}.

		\item \prefix{bd} est pour \whyprefix{b}{racket} \whyprefix{d}{ot} où \inenglish*{bracket} est pour \inenglish{crochet}.

		\medskip
		
		\item \prefix{f} est pour \whyprefix{f}{raction} avec aussi \prefix{sf} pour une écriture réduite où \prefix{s} est pour \whyprefix{s}{mall} soit \inenglish{petit}.

		\item \prefix{of} et \prefix{osf} utilisent le préfixe \prefix{o} pour \whyprefix{o}{pérateur}.
		
		\medskip
		
		\item \prefix{p} est pour \whyprefix{p}{arenthèse} : dans ce cas les parenthèses seront extensibles. Le fonctionnement est différent avec l'option \prefix{d} comme nous l'avons vu avant.

		\item \prefix{sp} est pour des parenthèses non extensibles. Là aussi le fonctionnement est différent avec l'option \prefix{d}.
	\end{enumerate}
\end{remark}


% ---------------------- %


\newparaexample{Pas de uns inutiles}

\begin{latexex}
 $\der[i ]{u}{x}{1}
= \der[f ]{u}{x}{1}
= \der[sf]{u}{x}{1}
= \der[of]{u}{x}{1}$
\end{latexex}


\begin{remark}
	Voici comment forcer les exposants $1$ si besoin. Fonctionnel mais très moche...

	\begin{latexex}
 $\der[i ]{u}{x}{\,\!1}
= \der[f ]{u}{x}{\,\!1}
= \der[sf]{u}{x}{\,\!1}
= \der[of]{u}{x}{\,\!1}$
\end{latexex}
\end{remark}

% ---------------------- %


\subsection{Fiches techniques}

\IDmacro{der}{1}{3}

\IDoption{} la valeur par défaut est \verb+u+. 
\begin{enumerate}
	\item \verb+u  + : écriture usuelle avec des primes \emph{(ceci nécessite d'avoir une valeur entière naturelle connue du nombre de dérivations successives)}.

	\item \verb+e  + : écriture via un exposant entre des parenthèses.
	
	\item \verb+i  + : écriture via un indice.

	\item \verb+d  + : écriture pointée à la physicienne \emph{(cf. la dérivation par rapport au temps)}.

	\item \verb+bd + : écriture pointée avec un crochet entre les points et la fonction.

	\medskip
	
	\item \verb+f  + : écriture via une fraction en mode display.

	\item \verb+sf + : écriture via une fraction en mode non display.

	\item \verb+of + : écriture via une fraction en mode display sous la forme d'un opérateur \emph{(la fonction est à côté de la fraction)}.

	\item \verb+osf+ : écriture via une fraction en mode non display sous la forme d'un opérateur \emph{(la fonction est à côté de la fraction)}.

	\medskip
	
	\item \verb+p  + : ajout de parenthèses extensibles autour de la fonction.

	\item \verb+sp + : ajout de parenthèses non extensibles autour de la fonction.
\end{enumerate}


\IDarg{1} la fonction à dériver.

\IDarg{2} la variable de dérivation.

\IDarg{3} l'ordre de dérivation.


% ---------------------- %


\subsection{Dérivations totales d'une fonction -- Version courte sans variable} \label{tnsana-short-der}

Dans l'exemple suivant le code manque de sémantique car on n'indique pas la variable de dérivation.
Ceci étant dit à l'usage la macro \macro{sder} rend de grands services.
Ici le préfixe \prefix{s} est pour \whyprefix{s}{imple} voire \whyprefix{s}{impliste}...
Voici des exemples où de nouveau l'option par défaut \prefix{u} et l'option \prefix{d} ne seront fonctionnelles qu'avec un ordre de dérivation de valeur naturelle connue !


\newparaexample{}

\begin{latexex}
 $\sder{f}{1} = \der{f}{x}{1}$

 $\sder   {f}{3}
= \sder[e]{f}{3}
= \sder[d]{f}{3}$
\end{latexex}


\newparaexample{}

\begin{latexex}
 $\sder[sp] {\dfrac{1}{2} uv}{2}
= \sder[e,p]{\dfrac{1}{2} uv}{2}
= \sder[bd] {\dfrac{1}{2} uv}{2}$
\end{latexex}


\begin{remark}
	Ici les seules options disponibles sont \prefix{u}, \prefix{e}, \prefix{b}, \prefix{bd}, \prefix{p} et \prefix{sp}.
\end{remark}


% ---------------------- %


\subsection{Fiches techniques}

\IDmacro{sder}{1}{2} où \quad \mwhyprefix{s}{imple}

\IDoption{} la valeur par défaut est \verb+u+. Les options disponibles sont \verb+u+, \verb+e+, \verb+d+, \verb+bd+, \verb+p+ et \verb+sp+ : voir la fiche technique de \macro{sder} ci-dessus.

\IDarg{1} la fonction à dériver.

\IDarg{2} l'ordre de dérivation.


% ---------------------- %


\subsection{L'opérateur de dérivation totale} \label{tnsana-ope-total-der}

Ce qui suit peut rendre service au niveau universitaire.
Les options possibles sont \verb+f+, valeur par défaut, \verb+sf+ et \verb+i+ avec les mêmes significations que pour la macro \macro{der}.

\begin{latexex}
 $\derope    {x}{k}
= \derope[sf]{x}{k}
= \derope[i] {x}{k}$
\end{latexex}


Ici non plus il n'y a pas de uns inutiles mais l'astuce \verb+\,\!1+ reste utilisable.

\begin{latexex}
 $\derope    {x}{1}
= \derope[sf]{x}{1}
= \derope[i] {x}{1}$
\end{latexex}


% ---------------------- %


\subsection{Fiches techniques}

\IDmacro{derope}{1}{2} où \quad \mwhyprefix{ope}{rator}

\IDoption{} la valeur par défaut est \verb+f+. Les options disponibles sont \verb+f+, \verb+sf+ et \verb+i+ : voir la fiche technique de \macro{der} donnée un peu plus haut.

\IDarg{1} la fonction à dériver.

\IDarg{2} l'ordre de dérivation.

\end{document}
